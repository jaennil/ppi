\documentclass[14pt]{extarticle}

\usepackage{fontspec}
\setmainfont{Times New Roman}

% размер полей
\usepackage{geometry}
\geometry{a4paper, top=2cm, bottom=2cm, right=1.5cm, left=3cm}

 %debugging
%\usepackage{showframe}

% полуторный интервал
\usepackage{setspace}
\onehalfspacing

% абзацный отступ
\setlength{\parindent}{1.25cm}

% выравнивание текста по ширине
\sloppy

% списки
\usepackage{calc} % арифметические операции с величинами
\usepackage{enumitem}
\setlist{
    nosep,
    leftmargin=0pt,
    itemindent=\parindent + \labelwidth - \labelsep,
}

% подписи к рисункам и таблицам
\usepackage{caption}
\renewcommand{\figurename}{Рисунок}
\renewcommand{\tablename}{Таблица}
\DeclareCaptionFormat{custom}
{
    \textit{#1#2#3}
}
\DeclareCaptionLabelSeparator{custom}{. }
\captionsetup{
    % хз какой это размер - 12 или нет, но выглядит меньше 14
    font=small,
    format=custom,
    labelsep=custom,
}

% картинки
\usepackage{graphicx}

% колонтитулы
\usepackage{fancyhdr}

% картинки и таблицы находятся именно в том месте текста где помещены (атрибут H)
\usepackage{float}

% таблицы
\usepackage{tabularray}

\graphicspath{ {2.2.1/models/} }
\begin{document}
\pagestyle{fancy}
\fancyhead{}
% disable header
\renewcommand{\headrulewidth}{0pt}
\fancyfoot[L]{Дубровских гр 221-361}
\fancyfoot[C]{ЛР 2.2.1}
\fancyfoot[R]{Продажа автотранспорта}
\singlespacing

\newpage
\begin{center}
    Министерство науки и высшего образования Российской Федерации
    Федеральное государственное автономное образовательное учреждение

    высшего образования

    \guillemotleft МОСКОВСКИЙ ПОЛИТЕХНИЧЕСКИЙ УНИВЕРСИТЕТ\guillemotright

    (МОСКОВСКИЙ ПОЛИТЕХ)
\end{center}
\noindent
\bigbreak
\bigbreak
\bigbreak
\bigbreak
\begin{center}
    ЛАБОРАТОРНАЯ РАБОТА 5.2.1

    По курсу Проектирования пользовательских интерфейсов в веб

    \textbf{Проектирование композиции и визуальной иерархии в макете веб-страниц и мобильного устройства}
    \bigbreak
    \bigbreak
    \bigbreak
    \bigbreak
    ТЕМА

    \guillemotleft\textbf{САЙТ ДЛЯ ПРОДАЖИ И ПОИСКА АВТОМОБИЛЕЙ}\guillemotright
\end{center}
\noindent
\bigbreak
\bigbreak
\bigbreak
\bigbreak
\bigbreak
\bigbreak
\bigbreak
\bigbreak
\bigbreak
\bigbreak
\hfill Выполнил

\hfill Дубровских Никита Евгеньевич

\hfill Группа 221-361
\bigbreak
\bigbreak
\bigbreak
\hfill Проверил

\hfill Натур ВВ
\vfill
\begin{center}
    Москва, 2024
\end{center}
\newpage
\onehalfspacing


\begin{center}
    \textbf{Лабораторная работа 2.2.1}

    \textbf{РАЗРАБОТКА ПОЛЬЗОВАТЕЛЬСКИХ СЦЕНАРИЕВ. СОСТАВЛЕНИЕ БЛОК-СХЕМ ТИПИЧНОГО СЦЕНАРИЯ}
\end{center}

\textbf{Цель работы:} придумать все возможные варианты взаимодействия пользователей и интерфейса.
\bigskip

\textbf{Задачи:}

\begin{enumerate}
    \item В виде блок-схем или в виде списка разработать возможные сценарии использования проектируемого интерфейса.
    \item Проанализировать полученные результаты и оптимизировать временные затраты (свести к минимуму количество шагов пользователя для достижения его целей).

\end{enumerate}
\bigskip

\textbf{Основные термины}

\begin{itemize}
    \item Пользовательские сценарии - наглядное схематическое представление того, как пользователь решает свою задачу с помощью сайта или приложения.
    \item Блок-схемы - графическое представление сценариев использования, показывающее последовательность действий пользователя.
    \item Сценарий использования (Use case) - описание поведения системы при взаимодействии с внешней средой.
    \item Пользовательские маршруты (User flows) - последовательность шагов, которые пользователь проходит для достижения своей цели на сайте или в приложении.
    \item Диаграмма потоков задач (Task flows) - визуальное представление шагов, необходимых для выполнения задачи пользователем.
    \item Оптимизация временных затрат - процесс минимизации количества шагов пользователя для достижения его целей.
    \item Варианты использования - список шагов, которые может пройти пользователь для достижения своей цели.
\end{itemize}
\bigskip

\textbf{Сценарии использования}

Сценарий 1: Поиск автомобиля

\begin{enumerate}
    \item Пользователь заходит на сайт.
    \item Пользователь видит главную страницу с поисковой строкой.
    \item Пользователь вводит марку и модель автомобиля в поисковую строку.
    \item Пользователь нажимает кнопку "Поиск".
    \item Система отображает список автомобилей, соответствующих запросу.
    \item Пользователь просматривает список и выбирает интересующий автомобиль.
    \item Пользователь переходит на страницу с подробной информацией об автомобиле.
    \item Пользователь может связаться с продавцом или сохранить объявление.
\end{enumerate}
\bigskip

Сценарий 2: Размещение объявления о продаже автомобиля

\begin{enumerate}
    \item Пользователь заходит на сайт.
    \item Пользователь нажимает кнопку "Подать объявление".
    \item Пользователь заполняет форму с информацией о автомобиле (марка, модель, год выпуска, цена, описание, фотографии).
    \item Пользователь нажимает кнопку "Опубликовать".
    \item Система подтверждает успешное размещение объявления.
    \item Пользователь получает уведомление о размещении объявления.
\end{enumerate}
\bigskip

Сценарий 3: Фильтрация результатов поиска

\begin{enumerate}
    \item Пользователь заходит на сайт.
    \item Пользователь вводит запрос в поисковую строку.
    \item Система отображает список автомобилей.
    \item Пользователь использует фильтры (цена, год выпуска, пробег, тип кузова) для уточнения поиска.
    \item Система обновляет список автомобилей в соответствии с выбранными фильтрами.
    \item Пользователь выбирает интересующий автомобиль и переходит на страницу с подробной информацией.
\end{enumerate}
\bigskip

Сценарий 4: Сравнение автомобилей

\begin{enumerate}
    \item Пользователь заходит на сайт.
    \item Пользователь вводит запрос в поисковой строке и получает список автомобилей.
    \item Пользователь выбирает несколько автомобилей для сравнения (например, отмечает галочками).
    \item Пользователь нажимает кнопку "Сравнить".
    \item Система отображает таблицу сравнения выбранных автомобилей по ключевым характеристикам (цена, пробег, год выпуска, состояние и т.д.).
    \item Пользователь анализирует сравнение и выбирает интересующий автомобиль для дальнейшего изучения.
\end{enumerate}
\bigskip

Сценарий 5: Подписка на уведомления

\begin{enumerate}
    \item Пользователь заходит на сайт.
    \item Пользователь вводит параметры поиска (марка, модель, цена).
    \item Пользователь нажимает кнопку "Подписаться на уведомления".
    \item Система запрашивает адрес электронной почты для отправки уведомлений.
    \item Пользователь вводит адрес электронной почты и подтверждает подписку.
    \item Система отправляет уведомления о новых объявлениях, соответствующих заданным параметрам.
\end{enumerate}
\bigskip

Сценарий 6: Оставление отзыва о продавце

\begin{enumerate}
    \item Пользователь заходит на сайт.
    \item Пользователь находит автомобиль, который купил, и переходит на страницу объявления.
    \item Пользователь нажимает кнопку "Оставить отзыв".
    \item Пользователь заполняет форму отзыва (оценка, текст отзыва).
    \item Пользователь нажимает кнопку "Отправить".
    \item Система подтверждает успешное размещение отзыва.
\end{enumerate}
\bigskip

\textbf{Анализ полученных результатов и оптимизация временных затрат}

Исходя из составленных вариантов использования можно сформировать следующие оптимизации временных затрат:

\begin{itemize}
    \item Упрощение навигации: главная страница содержит четкие и понятные ссылки на основные функции (поиск, размещение объявления, фильтры). Это поможет пользователю быстрее находить нужные разделы.
    \item Автозаполнение: внедрить функцию автозаполнения в поисковой строке, что может сократить время, необходимое для ввода данных, и уменьшить количество ошибок.
    \item Сохранение поиска: позволить пользователям сохранять свои поисковые запросы и фильтры, чтобы они могли быстро возвращаться к ним в будущем.
    \item Упрощение формы размещения объявления: форма для размещения объявления максимально проста и интуитивно понятна. Использовать подсказки и примеры, чтобы помочь пользователям заполнить форму быстрее.
    \item Оптимизация загрузки страниц: за счёт сжатия изображений, использование технологий кэширования страницы загружаются быстро, чтобы пользователи не теряли время на ожидание.
    \item Мобильная версия: удобный интерфейс для мобильных устройств, так как многие пользователи могут заходить на сайт с телефонов.
\end{itemize}
\bigskip

\textbf{Заключение}

Оптимизация пользовательских сценариев и минимизация шагов, необходимых для достижения целей, помогут улучшить пользовательский опыт и увеличить количество успешных взаимодействий на сайте.
\bigskip

\textbf{Контрольные вопросы и ответы}

\begin{enumerate}
    \item Что такое пользовательские сценарии и зачем они нужны?

    Пользовательские сценарии — это описания того, как пользователи взаимодействуют с продуктом (сайтом или приложением) для достижения своих целей. Они помогают понять потребности и поведение пользователей, а также выявить возможные проблемы в интерфейсе. Сценарии необходимы для проектирования удобных и эффективных интерфейсов, так как они позволяют разработчикам и дизайнерам сосредоточиться на реальных задачах пользователей.
    \item Что такое пользовательские маршруты (user flows) и зачем они нужны?

    Пользовательские маршруты (user flows) — это визуальные представления последовательности шагов, которые пользователь проходит для выполнения конкретной задачи в приложении или на сайте. Они помогают понять, как пользователи перемещаются по интерфейсу, какие действия выполняют и какие решения принимают. Пользовательские маршруты необходимы для оптимизации навигации и улучшения пользовательского опыта, позволяя выявить узкие места и улучшить взаимодействие.
    \item Что такое «диаграмма потоков задач» (Task flows)? На каком этапе и как ее строят?

    Диаграмма потоков задач (Task flows) — это графическое представление шагов, необходимых для выполнения конкретной задачи пользователем. Она строится на этапе проектирования, когда уже собраны данные о потребностях пользователей и определены основные сценарии использования. Для создания диаграммы потоков задач используются блоки, представляющие действия пользователя, и стрелки, показывающие направление потока. Это помогает визуализировать процесс и выявить возможные улучшения.
    \item Расскажите про основные элементы диаграмм потоков задач.

    Основные элементы диаграмм потоков задач включают:
    блоки действий - представляют собой шаги, которые пользователь должен выполнить (например, "Нажать кнопку", "Заполнить форму"),
    стрелки - показывают направление потока и последовательность действий,
    решения - точки, где пользователь может выбрать один из нескольких вариантов (например, "Да/Нет"),
    начало и конец - обозначают стартовую и конечную точки процесса,
    подпроцессы - могут быть использованы для обозначения более сложных действий, которые могут быть разбиты на отдельные шаги.
\end{enumerate}

\end{document}
