\documentclass[14pt]{extarticle}

\usepackage{fontspec}
\setmainfont{Times New Roman}

% размер полей
\usepackage{geometry}
\geometry{a4paper, top=2cm, bottom=2cm, right=1.5cm, left=3cm}

 %debugging
%\usepackage{showframe}

% полуторный интервал
\usepackage{setspace}
\onehalfspacing

% абзацный отступ
\setlength{\parindent}{1.25cm}

% выравнивание текста по ширине
\sloppy

% списки
\usepackage{calc} % арифметические операции с величинами
\usepackage{enumitem}
\setlist{
    nosep,
    leftmargin=0pt,
    itemindent=\parindent + \labelwidth - \labelsep,
}

% подписи к рисункам и таблицам
\usepackage{caption}
\renewcommand{\figurename}{Рисунок}
\renewcommand{\tablename}{Таблица}
\DeclareCaptionFormat{custom}
{
    \textit{#1#2#3}
}
\DeclareCaptionLabelSeparator{custom}{. }
\captionsetup{
    % хз какой это размер - 12 или нет, но выглядит меньше 14
    font=small,
    format=custom,
    labelsep=custom,
}

% картинки
\usepackage{graphicx}

% колонтитулы
\usepackage{fancyhdr}

% картинки и таблицы находятся именно в том месте текста где помещены (атрибут H)
\usepackage{float}

% таблицы
\usepackage{tabularray}

\graphicspath{ {8.2.1/models/} }
\usepackage{hyperref}
\begin{document}
\pagestyle{fancy}
\fancyhead{}
% disable header
\renewcommand{\headrulewidth}{0pt}
\fancyfoot[L]{Дубровских гр 221-361}
\fancyfoot[C]{ЛР 8.2.1}
\fancyfoot[R]{Продажа автотранспорта}
\singlespacing

\newpage
\begin{center}
    Министерство науки и высшего образования Российской Федерации
    Федеральное государственное автономное образовательное учреждение

    высшего образования

    \guillemotleft МОСКОВСКИЙ ПОЛИТЕХНИЧЕСКИЙ УНИВЕРСИТЕТ\guillemotright

    (МОСКОВСКИЙ ПОЛИТЕХ)
\end{center}
\noindent
\bigbreak
\bigbreak
\bigbreak
\bigbreak
\begin{center}
    ЛАБОРАТОРНАЯ РАБОТА 8.2.1

    По курсу Проектирования пользовательских интерфейсов в веб

    \textbf{Создание основных страниц (прототипов страниц) в программном ресурсе.Использование Гайдлайнов и UI- Kit}
    \bigbreak
    \bigbreak
    \bigbreak
    \bigbreak
    ТЕМА

    \guillemotleft\textbf{САЙТ ДЛЯ ПРОДАЖИ И ПОИСКА АВТОМОБИЛЕЙ}\guillemotright
\end{center}
\noindent
\bigbreak
\bigbreak
\bigbreak
\bigbreak
\bigbreak
\bigbreak
\bigbreak
\bigbreak
\bigbreak
\bigbreak
\hfill Выполнил

\hfill Дубровских Никита Евгеньевич

\hfill Группа 221-361
\bigbreak
\bigbreak
\bigbreak
\hfill Проверил

\hfill Натур ВВ
\vfill
\begin{center}
    Москва, 2024
\end{center}
\newpage
\onehalfspacing


\begin{center}
    \textbf{Лабораторная работа 8.2.1}

    \textbf{Создание основных страниц (прототипов страниц) в программном ресурсе.Использование Гайдлайнов и UI- Kit}
\end{center}

\textbf{Цель работы:} разработать макеты демонстрационных страниц для прототипа интерфейса веб-сайта (мобильного приложения)
\bigskip

\textbf{Задачи:}

\begin{enumerate}
    \item Используя принципы «атомарного дизайна», разработать все необходимые компоненты страниц
    \item Познакомиться с Дизайн-системой - UI- KIT, фреймворки и гайдлайны
    \item Рассмотреть и применить при проектировании гайдлайны
    \item Рассмотреть и применить при проектировании UI- Kit
    \item Собрать страницы из компонентов и необходимых графических элементов по ранее разработанным вайрфреймам
\end{enumerate}
\bigskip

\textbf{Основные термины}

\begin{itemize}
    \item Атомарный дизайн — методология, в которой интерфейс разделяется на минимальные функциональные единицы (атомы), такие как кнопки, иконки, чекбоксы и т.д. Эти элементы собираются в более сложные компоненты (молекулы, организмы) и системы для создания интерфейсов.
    \item Компоненты в Figma — повторяющиеся элементы дизайна, которые можно использовать на нескольких страницах (например, шапка или подвал сайта). При изменении мастер-компонента все дочерние копии обновляются автоматически.
    \item UI-kit (набор интерфейсных элементов) — это готовый комплект графических элементов для пользовательского интерфейса, который можно использовать для создания страниц и экранов. Он помогает стандартизировать дизайн, экономит время и улучшает командное взаимодействие.
    \item Гайдлайны — набор правил и рекомендаций по дизайну интерфейсов, который определяет, как должны выглядеть и взаимодействовать элементы приложения. Примеры популярных гайдлайнов — Material Design (Google) и Human Interface Guidelines (Apple).
    \item Респонсивная верстка — адаптация дизайна для разных размеров экранов с использованием гибких сеток, изображений и CSS-медиа-запросов.
    \item Адаптивная верстка — использование нескольких фиксированных макетов для различных разрешений экранов.
\end{itemize}
\bigskip

\textbf{Гайдлайн и UI-кит}
\bigskip

В качестве гайдлайна и UI-кита был выбран Material Design 3:

\begin{table}[H]
    \small
    \setstretch{1}
    \begin{tabular}{|p{3cm}|p{5cm}|p{6cm}|}
\hline
\textbf{Компонент} & \textbf{Описание} & \textbf{Рекомендации} \\
\hline
Цвет & Цветовая палитра Material Design включает основные и акцентные цвета. & Используйте яркие, насыщенные цвета для акцентов, приглушенные для фона. Цвета должны быть доступными для всех пользователей. \\
\hline
Типографика & Material использует набор шрифтов, в основном Roboto и Noto. & Стремитесь к ясности и читаемости. Размеры шрифтов от 12 до 96 px для различных уровней иерархии. \\
\hline
Иконки & Простые, монотонные, легко читаемые иконки с четким значением. & Размер иконок от 18 до 48 dp. Используйте Material Icons или их аналоги. \\
\hline
Отступы и сетка & Сетка и отступы помогают организовать пространство, создавая гармоничную структуру. & Стандартный отступ — 8 dp. Сетки 4 и 8 dp обеспечивают аккуратное расположение элементов. \\
\hline
Карточки & Карточки группируют информацию и действия для отдельной темы. & Используйте тени и закругленные углы. Размеры иерархичны и зависят от контекста. \\
\hline
Кнопки & Кнопки используют понятные действия и предлагают кликать по ним. & Основные кнопки – акцентные цвета, а второстепенные – приглушенные. Закруглённые углы улучшают восприятие. \\
\hline
Анимация и переходы & Анимации добавляют интерактивности и помогают пользователям понять результаты своих действий. & Анимации должны быть короткими (до 300 мс) и поддерживать естественность движений. \\
\hline
Взаимодействие & Касания, жесты и поведение взаимодействий стандартизированы. & Минимальный размер для кликабельных областей – 48x48 dp. Жесты должны быть интуитивными. \\
\hline
Поверхности и слои & Система поверхностей основана на Z-оси, что помогает организовать слои информации. & Используйте тени и высоту для обозначения слоев, акцентируя важные элементы. \\
\hline
Состояния элементов & Состояния (нажато, наведено, отключено) помогают пользователям понять текущую доступность или состояние элементов интерфейса. & Обозначайте состояния с помощью цветов и тени. Отключенные элементы должны выглядеть приглушённо и быть неподвижны. \\
\hline
Форма элементов & Закругленные углы и плавные переходы делают интерфейс более дружественным и современным. & Рекомендуется закругление от 4 до 16 dp в зависимости от компонента. \\
\hline
\end{tabular}
\end{table}

\noindent
\begin{minipage}{\linewidth}
    \fbox{\includegraphics[width=\linewidth]{m3_1}}
    \captionof{figure}{Пример UI-кит веб}
\end{minipage}
\bigskip

\noindent
\begin{minipage}{\linewidth}
    \fbox{\includegraphics[scale=0.55]{m3_2}}
    \captionof{figure}{Пример UI-кит мобильного приложения}
\end{minipage}
\bigskip

Ссылка:

\url{https://www.figma.com/community/file/1035203688168086460/material-3-design-kit}
\bigskip

\textbf{Прототипы}
\bigskip

На рисунках 3 – 7 представлен интерфейс спроектированного сайта. 
На  рисунке  3  представлена  главная  страница  сайта,  на  которой 
расположены карточки объявлений, в шапке есть кнопа входа.
При  нажатии  на  кнопку  “Войти”  пользователь  попадает  на  страницу 
авторизации (рисунок 4).
При нажатии на кнопку "Все предложения" пользователь попадает на страницу поиска, изображенную на рисунке 5.
Оттуда он может нажать на интересующее его объявление и попасть на его страницу, изображенную на рисунке 6.
В подвале сайта расположена карта сайта. При нажатии на кнопку "Контакты", пользователь попадает на страницу контактов, изображенную на рисунке 7.
\bigskip

\noindent
\begin{minipage}{\linewidth}
    \fbox{\includegraphics[width=\linewidth]{my1}}
    \captionof{figure}{Главная страница}
\end{minipage}
\bigskip

\noindent
\begin{minipage}{\linewidth}
    \fbox{\includegraphics[width=\linewidth]{my2}}
    \captionof{figure}{Страница авторизации}
\end{minipage}
\bigskip

\noindent
\begin{minipage}{\linewidth}
    \fbox{\includegraphics[width=\linewidth]{my3}}
    \captionof{figure}{Страница поиска}
\end{minipage}
\bigskip

\noindent
\begin{minipage}{\linewidth}
    \fbox{\includegraphics[width=\linewidth]{my4}}
    \captionof{figure}{Страница выбранного объявления}
\end{minipage}
\bigskip

\noindent
\begin{minipage}{\linewidth}
    \fbox{\includegraphics[width=\linewidth]{my5}}
    \captionof{figure}{Страница контактов}
\end{minipage}
\bigskip

Ссылка на Figma:

\url{https://www.figma.com/design/25STw8qqDFUKFnCIDQb1ri/PPI?node-id=0-1&t=3GTfYjQGpyVoN70i-1}
\bigskip

\textbf{Контрольные вопросы и ответы}

\begin{enumerate}
    \item Что такое атомарный дизайн веб-сайта?

    Атомарный дизайн — это методология создания интерфейсов, основанная на делении компонентов на более мелкие части. Она включает «атомы» (базовые элементы, такие как кнопки и ссылки), «молекулы» (группы элементов, которые работают вместе, как поле поиска), и «организмы» (большие группы элементов, например, шапка или карточка товара). Это помогает строить интерфейс системно и гибко, облегчая редактирование и стандартизацию.

\item Когда используются гайдлайны?

    Гайдлайны используются для обеспечения консистентности интерфейса и удобства взаимодействия пользователя с приложением или сайтом. Они предлагают стандарты, инструкции и рекомендации, как дизайн должен выглядеть и функционировать. Гайдлайны нужны для создания приложений, которые будут соответствовать требованиям платформ (например, Material Design для Android или HIG для iOS).

\item Когда используются UI-Kit?

    UI-Kit применяется при проектировании интерфейсов для ускорения процесса дизайна. Это набор готовых компонентов (кнопок, полей, иконок и пр.), которые можно использовать для сборки страниц или приложений. UI-Kit полезен для поддержания единого стиля и упрощения командной работы, поскольку все элементы собраны в одном месте и имеют стандартизированный вид.

\item Какие основные блоки входят в главную страницу?

        Шапку (логотип, меню навигации, контактные данные)

        Основной баннер или заголовок

        Преимущества или ключевые предложения

        Основной контент (новости, товары, статьи и т.п.)

        Футер (контакты, ссылки, политика конфиденциальности)

    \item В чем различие респонсивной и адаптивной верстки?

    Респонсивная верстка автоматически адаптирует интерфейс под любой экран за счет использования гибких сеток и элементов. Адаптивная верстка предполагает создание нескольких заранее определенных макетов под определенные размеры экранов, переключаясь между ними.

\item Рекомендации по выбору формата сайта:

    При выборе формата сайта необходимо учитывать целевую аудиторию и тип контента. Для сервисов и новостных сайтов важно выбрать респонсивный формат, чтобы контент адаптировался на любых устройствах. Для приложений с фиксированными требованиями можно использовать адаптивный формат с макетами для разных разрешений.


\end{enumerate}

\end{document}
