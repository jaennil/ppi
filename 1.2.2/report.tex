\documentclass[14pt]{extarticle}

\usepackage{fontspec}
\setmainfont{Times New Roman}

% размер полей
\usepackage{geometry}
\geometry{a4paper, top=2cm, bottom=2cm, right=1.5cm, left=3cm}

 %debugging
%\usepackage{showframe}

% полуторный интервал
\usepackage{setspace}
\onehalfspacing

% абзацный отступ
\setlength{\parindent}{1.25cm}

% выравнивание текста по ширине
\sloppy

% списки
\usepackage{calc} % арифметические операции с величинами
\usepackage{enumitem}
\setlist{
    nosep,
    leftmargin=0pt,
    itemindent=\parindent + \labelwidth - \labelsep,
}

% подписи к рисункам и таблицам
\usepackage{caption}
\renewcommand{\figurename}{Рисунок}
\renewcommand{\tablename}{Таблица}
\DeclareCaptionFormat{custom}
{
    \textit{#1#2#3}
}
\DeclareCaptionLabelSeparator{custom}{. }
\captionsetup{
    % хз какой это размер - 12 или нет, но выглядит меньше 14
    font=small,
    format=custom,
    labelsep=custom,
}

% картинки
\usepackage{graphicx}

% колонтитулы
\usepackage{fancyhdr}

% картинки и таблицы находятся именно в том месте текста где помещены (атрибут H)
\usepackage{float}

% таблицы
\usepackage{tabularray}

\graphicspath{ {1.2.2/models/} }
\begin{document}
\pagestyle{fancy}
\fancyhead{}
% disable header
\renewcommand{\headrulewidth}{0pt}
\fancyfoot[L]{Дубровских гр 221-361}
\fancyfoot[C]{ЛР 1.2.2}
\fancyfoot[R]{Продажа автотранспорта}
\singlespacing

\newpage
\begin{center}
    Министерство науки и высшего образования Российской Федерации
    Федеральное государственное автономное образовательное учреждение

    высшего образования

    \guillemotleft МОСКОВСКИЙ ПОЛИТЕХНИЧЕСКИЙ УНИВЕРСИТЕТ\guillemotright

    (МОСКОВСКИЙ ПОЛИТЕХ)
\end{center}
\noindent
\bigbreak
\bigbreak
\bigbreak
\bigbreak
\begin{center}
    ЛАБОРАТОРНАЯ РАБОТА 5.2.1

    По курсу Проектирования пользовательских интерфейсов в веб

    \textbf{Проектирование композиции и визуальной иерархии в макете веб-страниц и мобильного устройства}
    \bigbreak
    \bigbreak
    \bigbreak
    \bigbreak
    ТЕМА

    \guillemotleft\textbf{САЙТ ДЛЯ ПРОДАЖИ И ПОИСКА АВТОМОБИЛЕЙ}\guillemotright
\end{center}
\noindent
\bigbreak
\bigbreak
\bigbreak
\bigbreak
\bigbreak
\bigbreak
\bigbreak
\bigbreak
\bigbreak
\bigbreak
\hfill Выполнил

\hfill Дубровских Никита Евгеньевич

\hfill Группа 221-361
\bigbreak
\bigbreak
\bigbreak
\hfill Проверил

\hfill Натур ВВ
\vfill
\begin{center}
    Москва, 2024
\end{center}
\newpage
\onehalfspacing


\begin{center}
    \textbf{Лабораторная работа 1.2.2}

    \textbf{СОСТАВЛЕНИЕ ПЕРСОНАЖЕЙ И КЛЮЧЕВОЙ ФИГУРЫ}
\end{center}

\textbf{Цель работы:} на основе анализа целевой аудитории, данных о пользователях (персонажа) сформулировать их цели использования веб-ресурса и разработать под них функционал интерфейса.
\bigskip

\textbf{Задачи:}

\begin{enumerate}
    \item Определить целевую аудиторию (группу) пользователей сайта 
    \item Определить основных участников (пользователей, владельцев, клиентов и т.п.), которые могут повлиять на решение задач по созданию веб-продукта
    \item Проанализировать данные о пользователях, об их предпочтениях, образе жизни, привычках, профессии
    \item Разработать ключевую персону (персонаж), дать характеристики, ее цели и потребности использования веб-ресурса
    \item Определить и зафиксировать, как эти данные (характеристики, цели и потребности) персоны будут использованы в проектировании интерфейса
\end{enumerate}
\bigskip

\textbf{Основные термины}

\begin{itemize}
    \item Целевая аудитория - группа пользователей, для которой разрабатывается веб-ресурс.
    \item Персонаж (персона) - собирательный образ пользователя, представляющий сегмент целевой аудитории.
    \item Ключевая персона - персонаж, для которого происходит проектирование информационной системы.
    \item Второстепенная персона - персонаж, чьи потребности учитываются, но не являются приоритетными.
    \item Цели использования веб-ресурса - задачи, которые пользователи хотят решить с помощью веб-ресурса.
    \item Функционал интерфейса - набор возможностей и функций, предоставляемых пользователю в веб-ресурсе.
    \item Демографические характеристики - пол, возраст, семейное положение, уровень образования и т.д.
    \item Психографические характеристики - привычки, интересы, мотивация и потребности пользователей.
    \item Контекст использования - среда, в которой пользователь взаимодействует с веб-ресурсом (рабочее место, устройство и т.д.).
    \item Методы исследования пользовательской аудитории - способы сбора данных о пользователях (анкеты, интервью, веб-аналитика и т.д.).
    \item Профилирование пользователей - процесс создания образа целевой аудитории на основе собранных данных.
    \item Пользовательская история (user story) - описание взаимодействия персонажа с веб-ресурсом, включая его цели и задачи.
    \item Исследование пользователей - процесс анализа данных о пользователях для понимания их потребностей и предпочтений.
\end{itemize}
\bigskip

\textbf{Определение целевой группы пользователей сайта.}

\begin{enumerate}
    \item Пол: преимущественно мужчины
    \item Возраст: 25-55 лет
    \item Семейное положение: одинокие, состоящие в отношениях, семейные
    \item Уровень владения компьютером: средний – пользователи уверенно работают с интернетом и основными приложениями
    \item Частота пользования интернетом: часто – каждый день несколько часов
    \item Уровень образования: высшее или среднее специальное
    \item Уровень оборудования: средний – ноутбук или настольный компьютер с хорошими характеристиками
    \item Тип соединения с интернетом: проводной или мобильный (Wi-Fi, 4G, Ethernet)
\end{enumerate}
\bigskip

Дополнительные пункты, важные для создания образа:
\begin{itemize}
    \item Интерес к автомобилям и технологиям
    \item Наличие опыта покупки или продажи автомобилей
    \item Открытость к использованию онлайн-сервисов для поиска и покупки автомобилей
\end{itemize}
\bigskip

Анализируя целевую аудиторию, можно сделать следующие выводы:
\begin{enumerate}
    \item Технологическая грамотность.

    Пользователи имеют средний уровень владения компьютером и активно используют интернет. Это предполагает, что сайт должен быть интуитивно понятным, с простым и удобным интерфейсом, а также адаптированным для мобильных устройств.

    \item Семейные ценности.

    Семейное положение пользователей (женатые, с детьми) указывает на то, что они могут искать автомобили, подходящие для семейных нужд. Это может включать в себя акцент на безопасность, вместимость и экономичность автомобилей.

    \item Интерес к технологиям.

    Пользователи заинтересованы в современных технологиях и автомобилях, что открывает возможности для внедрения дополнительных функций на сайте, таких как сравнение характеристик автомобилей.

    \item Частота использования.

    Пользователи часто обращаются к интернету для поиска информации, что подразумевает необходимость в актуальных и свежих данных на сайте. Регулярное обновление информации о предложениях и ценах будет критически важным.

    \item Ожидания от сайта.

    Пользователи ожидают удобного и быстрого доступа к информации, возможности фильтрации по различным параметрам и наличия отзывов. Это требует от разработчиков создания функционала, который будет удовлетворять эти потребности.

    \item Влияние окружения.

    Поскольку пользователи могут учитывать мнения друзей и коллег, важно создать платформу, которая будет способствовать обсуждению и обмену мнениями, например, через отзывы и рейтинги.

    \item Потребность в поддержке.

    Пользователи могут нуждаться в помощи при выборе автомобиля, что открывает возможности для внедрения консультационных услуг или чата с менеджерами.

    \item Фокус на ценности.

    Поскольку целевая аудитория ищет выгодные предложения, важно акцентировать внимание на ценовой политике, акциях и специальных предложениях.
\end{enumerate}
\bigskip

\textbf{Создание профиля пользователя (покупателя).}

\noindent
\begin{minipage}{\linewidth}
    \fbox{\includegraphics[scale=0.15]{it_guy}}
\end{minipage}
\bigskip

\begin{itemize}
    \item Имя: Алексей Смирнов
    \item Возраст: 32 года
    \item Пол: мужской
    \item Профессия: менеджер по продажам в IT-компании
    \item Внешность: среднего роста, спортивного телосложения, носит очки, предпочитает деловой стиль одежды
    \item Личностные характеристики: ответственный, целеустремленный, любит технологии и автомобили, активно использует интернет для поиска информации
    \item Семейное положение: женат, есть один ребенок
\end{itemize}
\bigskip

Цели персонажа:

\begin{itemize}
    \item Не связанные с информационной системой: найти надежный и экономичный автомобиль для семьи, который будет удобен для поездок по городу и на дачу.

    \item Связанные с информационной системой: можно разделить на личные и навязанные.
            Личные: найти и купить автомобиль по выгодной цене, сравнить предложения, изучить отзывы.
            Навязанные: учитывать рекомендации друзей и коллег, следовать трендам на рынке автомобилей.
\end{itemize}
\bigskip

Взаимодействие с продуктом:

\begin{itemize}
    \item Обстановка использования: дома или в офисе, за компьютером или с мобильного устройства.
    \item Частота использования: регулярно, несколько раз в неделю, особенно в выходные.
    \item Опыт использования подобных систем: имеет опыт использования сайтов для поиска и покупки автомобилей, знает, как фильтровать и сравнивать предложения.
    \item Факторы оценки информационной системы: удобство интерфейса, наличие актуальной информации, возможность фильтрации по параметрам, наличие отзывов.
    \item Ожидания от поведения и содержания информационной системы: быстрый доступ к информации, простота навигации, наличие актуальных предложений.
    \item Ожидаемые результаты от взаимодействия: найти подходящий автомобиль, получить полную информацию о нем, связаться с продавцом.
\end{itemize}
\bigskip

\textbf{Ключевой и второстепенный персонажи.}

Ключевой персонаж: Алексей Смирнов – основной пользователь, для которого будет разрабатываться сайт.

Второстепенный персонаж: Ольга Петрова, 28 лет, маркетолог, ищет первый автомобиль. Ее потребности также важны, но сайт будет в первую очередь ориентирован на Алексея.
\bigskip

\textbf{Контрольные вопросы и ответы}

\begin{enumerate}
\item Исследование и разработка целевой аудитории выбранной тематике веб-ресурса. Основные этапы исследования пользователей. Варианты исследования пользователей.
    \bigskip

Основные этапы исследования пользователей:

        \begin{itemize}
            \item Определение целей исследования: Четкое понимание, что именно нужно узнать о пользователях.
            \item Сбор данных: Использование различных методов для получения информации о пользователях.
            \item Анализ данных: Обработка собранной информации для выявления паттернов и характеристик целевой аудитории.
            \item Создание профилей пользователей: Формирование персонажей на основе анализа данных.
            \item Тестирование и валидация: Проверка созданных профилей на реальных пользователях для уточнения и корректировки.
        \end{itemize}
    \bigskip

Варианты исследования пользователей:

        \begin{itemize}
            \item Опросы и анкеты: Сбор количественных данных о пользователях.
            \item Интервью: Глубокое изучение мнений и потребностей пользователей.
            \item Фокус-группы: Обсуждение с группой пользователей для выявления их мнений и предпочтений.
            \item Анализ веб-аналитики: Изучение поведения пользователей на сайте.
            \item Наблюдение: Прямое наблюдение за поведением пользователей в реальных условиях.
        \end{itemize}
\item Какие задачи решает создание профиля пользователя (персоны, покупателя)?

    \begin{itemize}
        \item Понимание потребностей: Определение, что именно нужно пользователям.
        \item Улучшение пользовательского опыта: Создание более удобного и интуитивного интерфейса.
        \item Сегментация аудитории: Выделение различных групп пользователей для более точного таргетинга.
        \item Оптимизация контента: Адаптация контента под интересы и потребности целевой аудитории.
        \item Упрощение процесса разработки: Четкое понимание, для кого разрабатывается продукт, что упрощает принятие решений.
    \end{itemize}

\item Почему важно понять, кто является потенциальными пользователями (клиентами) сайта (приложения)?

    \begin{itemize}
        \item Увеличение эффективности: Знание целевой аудитории позволяет создавать более релевантный контент и функционал.
        \item Снижение рисков: Понимание потребностей пользователей помогает избежать ошибок в разработке.
        \item Улучшение маркетинга: Позволяет более точно нацеливать рекламные кампании и повышать их эффективность.
        \item Повышение удовлетворенности: Удовлетворение потребностей пользователей ведет к повышению их лояльности и удовлетворенности.
    \end{itemize}

\item Каковы методы исследования пользовательской аудитории?

    \begin{itemize}
        \item Качественные методы: Интервью, фокус-группы, наблюдение.
        \item Количественные методы: Опросы, анкетирование, анализ статистики.
        \item Анализ данных: Использование веб-аналитики и других инструментов для изучения поведения пользователей.
    \end{itemize}

\item Каковы источники сбора информации о пользователях?

    \begin{itemize}
        \item Веб-аналитика: Данные о поведении пользователей на сайте.
        \item Социальные сети: Анализ групп и профилей, обсуждений и комментариев.
        \item Обратная связь от клиентов: Отзывы, комментарии, обращения в службу поддержки.
        \item Исследования и отчеты: Данные из исследований рынка и отраслевых отчетов.
    \end{itemize}

\item Где искать данные для профилирования?
    \begin{itemize}
        \item Анализ текущих пользователей: Использование данных о существующих клиентах.
        \item Социальные сети: Изучение интересов и поведения пользователей в социальных сетях.
        \item Конкуренты: Анализ целевой аудитории конкурентов и их подходов.
        \item Опросы и интервью: Прямой сбор информации от пользователей.
    \end{itemize}

\item Как в проектировании используется информация о ключевой фигуре?
    \begin{itemize}
        \item Фокус на потребностях: Проектирование интерфейса и функционала с учетом потребностей ключевого персонажа.
        \item Сценарии использования: Разработка сценариев взаимодействия, основанных на поведении ключевого персонажа.
        \item Тестирование: Проверка прототипов и решений на соответствие ожиданиям ключевого персонажа.
        \item Адаптация контента: Создание контента, который будет наиболее интересен и полезен для ключевого персонажа.
    \end{itemize}
\end{enumerate}

\begin{enumerate}
\end{enumerate}

\end{document}
